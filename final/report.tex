%%% Template originaly created by Karol Kozioł (mail@karol-koziol.net) and modified for ShareLaTeX use

\documentclass[a4paper,11pt]{article}

\usepackage[T1]{fontenc}
\usepackage[utf8]{inputenc}
\usepackage{graphicx}
\usepackage{xcolor}

\usepackage{tgtermes}

\usepackage[
pdftitle={02247 Compiler Construction},
pdfauthor={Matthias Larsen, s103437},
colorlinks=true,linkcolor=blue,urlcolor=blue,citecolor=blue,bookmarks=true,
bookmarksopenlevel=2]{hyperref}
\usepackage{amsmath,amssymb,amsthm,textcomp}
\usepackage{enumerate}
\usepackage{multicol}
\usepackage{tikz}

\usepackage{geometry}
\geometry{total={210mm,297mm},
left=25mm,right=25mm,%
bindingoffset=0mm, top=20mm,bottom=20mm}


\linespread{1.3}

\newcommand{\linia}{\rule{\linewidth}{0.5pt}}

% custom theorems if needed
\newtheoremstyle{mytheor}
    {1ex}{1ex}{\normalfont}{0pt}{\scshape}{.}{1ex}
    {{\thmname{#1 }}{\thmnumber{#2}}{\thmnote{ (#3)}}}

\theoremstyle{mytheor}
\newtheorem{defi}{Definition}

% my own titles
\makeatletter
\renewcommand{\maketitle}{
\begin{center}
\vspace{2ex}
{\huge \textsc{\@title}}
\vspace{1ex}
\\
\linia\\
\@author \hfill \@date
\vspace{4ex}
\end{center}
}
\makeatother
%%%

% custom footers and headers
\usepackage{fancyhdr,lastpage}
\pagestyle{fancy}
\lhead{}
\chead{}
\rhead{}
\lfoot{Assignment \textnumero{} 3}
\cfoot{}
\rfoot{Page \thepage\ /\ \pageref*{LastPage}}
\renewcommand{\headrulewidth}{0pt}
\renewcommand{\footrulewidth}{0pt}
%

%%%----------%%%----------%%%----------%%%----------%%%

\begin{document}

\title{02247 Compiler Construction}

\author{Matthias Larsen, s103437}

\date{27/04/2016}

\maketitle


% Notes from Probst:
% * Main topic: copy/paste detection
% * * Describe what you implemented, what worked, what did not
% * * Describe how your compiler interacts with the intermediate representation
% * * Describe how you actually identify duplicates or likely duplicates
% * * DO NOT INCLUDE THE COMPLETE SOURCE CODE IN THE REPORT
% ​* * * Use snippets to show interesting parts!
% * * * Upload source code archive with the report
% * Equally important
% * * Structure of the copy/paste detector
% * * Integration with LLVM
% * * What did you learn in the project?


% Notes from TA:
% For the final report I would strongly suggest that you work on improving the report:

% - Make sure you describe in a lot more detail how the pass work:
% What are you trying to detect?
% How to you achive this?
% Break down the pass into the smaller steps that are performed to detect the duplication.

% The flow of the program is a little unintuitive, but actually a clever use of the LLVM API: use the function class to get all the call sites and then compare those directly.
% This is a fine approach, but make sure you describe how it works in the report.
% - Show examples (it's fine to show both the c source code and the IR you are working on) and describe what the program is using from that information in order to detect duplicates.

% - The future work section should be linked to the description of the pass:
% what would you want to change/improve? how and why?

% In short: you have a rudimentary pass that appear to be working - keep that and focus all your energy in the project on writing a good report and documenting what you actually achieved with the project.
% Don't sell you efforts short by only writing 8 lines about what you built :-) /Rasmus


\section*{Introduction}

\section*{Implementation}

\subsection*{First attempt}
My first attempt at a parser was very crude.
I started with a naive approach as a FunctionPass, thus limiting the scope of the copy/paste detection to a single method at a time.

As a starting point, the initial idea is to check every line of the program against the rest of the program, to see if it is identical, similar or not at all.

The LLVM framework provides two methods that can assist with this; isIdenticalTo and isSimilarTo.
The first one checks if the instruction is 100\% identical, where as the latter checks if the method call is the same.

The isSimilarTo part would then have to be checked further in order to see if its a possble error.

It's also the method we have to use in order for us to check for if a method has been pasted wrongly and should be identical to another instruction.

I chose to use a FunctionPass as I based my initial implementation on the SkeletonPass and thought to start solve the problem at a function level before expanding to a whole file / project.

This version of the pass could take the following code as input
\begin{verbatim}
#include <stdio.h>
int main( )
{
  printf("Hello World!\n");
  printf("Hello World!\n");
  if (1) {
    printf("Hello World 2!\n");
    printf("Hello World!\n");
  }
}
\end{verbatim}

and would produce the folling warnings
\begin{verbatim}
Identical method found. Possbile copy/paste
Original call: printf @ ../../test3.c:4, col 3
  Possble paste error: printf @ ../../test3.c:5, col 3
  Same operation found. Needs deeper check
  Possble paste error: printf @ ../../test3.c:7, col 4
  Possble paste error: printf @ ../../test3.c:8, col 4
\end{verbatim}

\subsection*{Second attempt}
I realised that my first attempt was very naive and would result in extremely long running times if the code was big enough.
After reading a bit about the LLVM API, I found that by exploiting some of the LLVM methods I could increase the effeciency of the Pass by somewhat comprising the flow, making it a bit more unintuitive.

\subsubsection*{Finding possible duplicates}
Instead of checking one line against the rest of the lines in the function (in case of FunctionPass), I found that I could use the function class to get all the call sites and compare those with each other instead, thus drastically reducing the time complexity of the Pass.



\subsubsection*{Comparing instructions}



\section*{Future work}
I had a few difficulties getting the variable name from a load instruction such that I could compare function arguments in `add(a, a)' vs `add(a, b)'. This section is my thoughts on how I'd extend the pass, had I time.

\subsection*{Improving detection}
Handling arithmic expressions (such as a+b and a*a) could be solved by creating a method that could take a Value as input and convert it to a string representation -- this would make it easier to handle complex cases than the current code.
That is, to convert `%5 = add nsw i32 %3, %4, !dbg !22` to `a+a` so that the program could make a string comparison for literals.
Doing so would allow checking complex arguments with nested arithmic operations as well.

Pseudo code:

\begin{verbatim}
processExpression(value):
  sequence <- initialize
  visit(sequence, value)
  generateSprintfCallFromSequence(sequence)

visit(sequence, value):
  if value is load:
    sequence.add(load.pointer)
  else if value is binaryop:
    sequence.add(openingParen)
    visit(sequence, binaryop.operand(0))
    sequence.add(binaryop.opcode)
    visit(sequence, binaryop.operand(1))
    sequence.add(closingParen)
\end{verbatim}

Adding the opening and closing paranthesis would be needed to distinguish `(b+c-d)*e` from `b+(c-d)*e`.

\subsection*{Detecting blocks}
Detecting blocks can be done in two ways.
The first is to introduce some ``lookahead'' functionality to the current pass. That is, to increment both pointers (I and Inst) as long as they're the same and then trigger a message when encountering a line that is just similar or way off.

The other way is to encode or serialize parts of the code as a graph and compare graphs to find blocks of duplicate code even though graph isomorphism is a hard problem.


\section*{Conclusion}

\end{document}